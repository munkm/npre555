% use the answers clause to get answers to print; otherwise leave it out.
\documentclass[12pts,answers,addpoints]{exam}
%\documentclass[12pts]{exam}
\RequirePackage{amssymb, amsfonts, amsmath, latexsym, verbatim, xspace, setspace, mathrsfs}
\usepackage{graphicx}

% By default LaTeX uses large margins.  This doesn't work well on exams; problems
% end up in the "middle" of the page, reducing the amount of space for students
% to work on them.
\usepackage[margin=1in]{geometry}
\usepackage{enumerate}
\usepackage[hidelinks]{hyperref}

% Here's where you edit the Class, Exam, Date, etc.
\newcommand{\class}{NPRE 555}
\newcommand{\term}{Fall 2023}
\newcommand{\assignment}{Homework 1}
\newcommand{\duedate}{2023.08.28}
%\newcommand{\timelimit}{50 Minutes}

\newcommand{\nth}{n\ensuremath{^{\text{th}}} }
\newcommand{\ve}[1]{\ensuremath{\mathbf{#1}}}
\newcommand{\Macro}{\ensuremath{\Sigma}}
\newcommand{\vOmega}{\ensuremath{\hat{\Omega}}}

% For an exam, single spacing is most appropriate
\singlespacing
% \onehalfspacing
% \doublespacing

% For an exam, we generally want to turn off paragraph indentation
\parindent 0ex

%\unframedsolutions

\begin{document}

% These commands set up the running header on the top of the exam pages
\pagestyle{head}
\firstpageheader{}{}{\makebox[0.5\textwidth]{\hfill Name: \underline{ Your Name }}}
\runningheader{\class}{\assignment\ - Page \thepage\ of \numpages}{Due \duedate}
\runningheadrule

\class \hfill \term \\
\assignment \hfill Due \duedate\\
\rule[1ex]{\textwidth}{.1pt}
%\hrulefill

%%%%%%%%%%%%%%%%%%%%%%%%%%%%%%%%%%%%%%%%%%%%%%%%%%%%%%%%%%%%%%%%%%%%%%%%%%%%%%%%%%%%%
%%%%%%%%%%%%%%%%%%%%%%%%%%%%%%%%%%%%%%%%%%%%%%%%%%%%%%%%%%%%%%%%%%%%%%%%%%%%%%%%%%%%%
\begin{itemize}
        \item Show your work.
        \item This work must be submitted online as a \textbf{single}
                \texttt{.pdf} file through gradescope.
\end{itemize}
\rule[1ex]{\textwidth}{.1pt}

% ---------------------------------------------
\begin{questions}
        \question In this class Dr. Munk will call on you randomly.
        \begin{parts}
          \part What name would you like Dr. Munk to use when she calls on you?
          Please include pronunciation directions if you think it would be
          helpful.
          \part What pronouns should she use for you in class?
          \part Is there anything you would like Dr. Munk to know about you in
          advance of the course?
        \end{parts}
        \question Git and github will be important tools for you to learn during
        this course.
        \begin{parts}
          \part What is your github username? Please add a link to your
        profile.
          \part Set up and configure git on your local machine as described in Ch15 of \textit{Effective
                Computation in Physics} by Scopatz and Huff. This text is available
                digitally through the UIUC library.
          \part Familiarize yourself with local and remote version control by
               reading and follow along with the guidance provided by Chs15
               and 16 of the aforementioned text. Make sure to set up ssh keys
               to push and pull from your github profile.
        \end{parts}
        \question Where is the instructor's office?
        \question Your TA will be an invaluable resource for this course
        \begin{parts}
                \part Who is your TA for this course?
        \end{parts}
        \question If you need extra help in the class, what are your options?
        \question Will late work be accepted?
        \question When will the lectures be held over zoom? How will you know?

\end{questions}



%\bibliographystyle{plain}
%\bibliography{hw01}
\end{document}
