% use the answers clause to get answers to print; otherwise leave it out.
\documentclass[12pts,answers,addpoints]{exam}
%\documentclass[12pts]{exam}
\RequirePackage{amssymb, amsfonts, amsmath, latexsym, verbatim, xspace, setspace, mathrsfs}
\usepackage{graphicx}

% By default LaTeX uses large margins.  This doesn't work well on exams; problems
% end up in the "middle" of the page, reducing the amount of space for students
% to work on them.
\usepackage[margin=1in]{geometry}
\usepackage{enumerate}
\usepackage[hidelinks]{hyperref}

% Here's where you edit the Class, Exam, Date, etc.
\newcommand{\class}{NPRE 555}
\newcommand{\term}{Fall 2023}
\newcommand{\assignment}{Homework 4}
\newcommand{\duedate}{2023.10.18}
%\newcommand{\timelimit}{50 Minutes}

\newcommand{\nth}{n\ensuremath{^{\text{th}}} }
\newcommand{\ve}[1]{\ensuremath{\mathbf{#1}}}
\newcommand{\Macro}{\ensuremath{\Sigma}}
\newcommand{\vOmega}{\ensuremath{\hat{\Omega}}}

% For an exam, single spacing is most appropriate
\singlespacing
% \onehalfspacing
% \doublespacing

% For an exam, we generally want to turn off paragraph indentation
\parindent 0ex

%\unframedsolutions

\begin{document}

% These commands set up the running header on the top of the exam pages
\pagestyle{head}
\firstpageheader{}{}{\makebox[0.5\textwidth]{\hfill Name: \underline{ Your Name }}}
\runningheader{\class}{\assignment\ - Page \thepage\ of \numpages}{Due \duedate}
\runningheadrule

\class \hfill \term \\
\assignment \hfill Due \duedate\\
\rule[1ex]{\textwidth}{.1pt}
%\hrulefill

%%%%%%%%%%%%%%%%%%%%%%%%%%%%%%%%%%%%%%%%%%%%%%%%%%%%%%%%%%%%%%%%%%%%%%%%%%%%%%%%%%%%%
%%%%%%%%%%%%%%%%%%%%%%%%%%%%%%%%%%%%%%%%%%%%%%%%%%%%%%%%%%%%%%%%%%%%%%%%%%%%%%%%%%%%%
\begin{itemize}
        \item Show your work.
        \item This work must be submitted online as a \textbf{single}
                \texttt{.pdf} file through gradescope.
        \item If you worked together with another student in the course, please
          document who you worked with and on what.
        \item If you used a numerical program (such as Python, Wolfram Alpha,
          etc.), all scripts must be submitted in addition to the \texttt{.pdf}.
          You may submit these via email to Sun Myung and myself.
\end{itemize}
\rule[1ex]{\textwidth}{.1pt}

% ---------------------------------------------
\begin{questions}
        % ---------------------------------------------
        \question Perform the following integrations.
        \begin{parts}
          \part[10] $\int_{-1}^{1} P_3(\mu)P_3(\mu)d\mu$

          \part[10] $\int_{-1}^{1} P_2(\mu)P_5(\mu)d\mu$
        \end{parts}

        \question[40] Given the following angular flux:
        \begin{align}
                        \psi(\mu) =
                \begin{cases}
                        \phi_0 & 0\le \mu \le 1\\
                        \phi_0  + a \mu^4 & -1\le \mu \le 0
                \end{cases}
        \end{align}
        expand using the $P_1$, and $P_3$ approximations. Plot your results on a single graph for
comparison.

        % ---------------------------------------------
        \question[20] Show that the $n=2$ order Legendre polynomial expansion of
        the transport equation leads to a second order equation (diffusion type
        equation). Compare this equation to the $n=1$ second order equation.
        Which quantity has changed? Does this represent a significant
        improvement over the $n=1$ second order equation?

        % ---------------------------------------------
        \question[20] Derive the P1 and P3 equations in spherical geometry starting from the time-independent, one-speed transport equation. The following recurrence relations are required:

        \begin{align}
                xP_n(x) = \frac{1}{2n+1}\left[(n+1)P_{n+1}(x) + n
                P_{n-1}(x)\right]\\
                (x^2 -1) \frac{dP_n}{dx} = n (xP_n(x)  - P_{n-1}(x))
        \end{align}
\end{questions}



%\bibliographystyle{plain}
%\bibliography{hw01}
\end{document}
