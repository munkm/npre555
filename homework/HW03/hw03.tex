% use the answers clause to get answers to print; otherwise leave it out.
\documentclass[12pts,answers,addpoints]{exam}
%\documentclass[12pts]{exam}
\RequirePackage{amssymb, amsfonts, amsmath, latexsym, verbatim, xspace, setspace, mathrsfs}
\usepackage{graphicx}

% By default LaTeX uses large margins.  This doesn't work well on exams; problems
% end up in the "middle" of the page, reducing the amount of space for students
% to work on them.
\usepackage[margin=1in]{geometry}
\usepackage{enumerate}
\usepackage[hidelinks]{hyperref}

% Here's where you edit the Class, Exam, Date, etc.
\newcommand{\class}{NPRE 555}
\newcommand{\term}{Fall 2023}
\newcommand{\assignment}{Homework 3}
\newcommand{\duedate}{2023.10.04}
%\newcommand{\timelimit}{50 Minutes}

\newcommand{\nth}{n\ensuremath{^{\text{th}}} }
\newcommand{\ve}[1]{\ensuremath{\mathbf{#1}}}
\newcommand{\Macro}{\ensuremath{\Sigma}}
\newcommand{\vOmega}{\ensuremath{\hat{\Omega}}}

% For an exam, single spacing is most appropriate
\singlespacing
% \onehalfspacing
% \doublespacing

% For an exam, we generally want to turn off paragraph indentation
\parindent 0ex

%\unframedsolutions

\begin{document}

% These commands set up the running header on the top of the exam pages
\pagestyle{head}
\firstpageheader{}{}{\makebox[0.5\textwidth]{\hfill Name: \underline{ Your Name }}}
\runningheader{\class}{\assignment\ - Page \thepage\ of \numpages}{Due \duedate}
\runningheadrule

\class \hfill \term \\
\assignment \hfill Due \duedate\\
\rule[1ex]{\textwidth}{.1pt}
%\hrulefill

%%%%%%%%%%%%%%%%%%%%%%%%%%%%%%%%%%%%%%%%%%%%%%%%%%%%%%%%%%%%%%%%%%%%%%%%%%%%%%%%%%%%%
%%%%%%%%%%%%%%%%%%%%%%%%%%%%%%%%%%%%%%%%%%%%%%%%%%%%%%%%%%%%%%%%%%%%%%%%%%%%%%%%%%%%%
\begin{itemize}
        \item Show your work.
        \item This work must be submitted online as a \textbf{single}
                \texttt{.pdf} file through gradescope.
        \item If you worked together with another student in the course, please
          document who you worked with and on what.
        \item If you used a numerical program (such as Python, Wolfram Alpha,
          etc.), all scripts must be submitted in addition to the \texttt{.pdf}.
          You may submit these via email to Sun Myung and myself.
\end{itemize}
\rule[1ex]{\textwidth}{.1pt}

% ---------------------------------------------
\begin{questions}
        % ---------------------------------------------
        \question[20]
        If X is a continuous random variable, show the following relation is
        true :
        \begin{align*}
                Var(aX + b) = a^2Var(X)
        \end{align*}

        where $Var()$ is the variance.

        % ---------------------------------------------
        \question[20]
        (Stacey 9.13)
        The pdf for variable $x$ is
        $f(s)=\frac{4}{\pi\left(1+x^2\right)}$ with $0\le x \le 1$. Show that if
        a random number $\xi$ between zero and one $\left(0 \le \xi \le
        1\right)$ is generated, the corresponding value of
        $x=\tan\left(\frac{\xi\pi}{4}\right)$.

        % ---------------------------------------------
        \question[30]
        (Stacey 9.17)
        Plot the cumulative distribution function
        corresponding to the fission spectrum given approximately by:

        \begin{align*}
                \chi(E) = 0.453 e^{-1.036E}\sinh\sqrt{2.29E}
                \intertext{where}
                10^4 eV \le E \le 10^7 eV
        \end{align*}

        % ---------------------------------------------
        \question[30]
        (Stacey 9.19)
        Plot the pdf and the cdf for the cross section
        distribution in a region with
        $\Sigma_a = 0.15$ cm$^{-1}$,
        $\Sigma_s = 0.08$ cm$^{-1}$, and
        $\Sigma_f = 0.08$ cm$^{-1}$.

\end{questions}



%\bibliographystyle{plain}
%\bibliography{hw01}
\end{document}
